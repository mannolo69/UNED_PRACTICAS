\documentclass{article}
\usepackage{amsmath} % for the equation* environment
\usepackage[spanish]{babel}
\usepackage{datetime}
\usepackage{parskip}
\usepackage{graphicx}
\usepackage{xcolor}
\usepackage{lineno}
%\usepackage[latin1]{inputenc}
\title{Práctica 1 }
\author{Manuel Díaz Castro}
\date{04 de Marzo de 2024}
\setlength{\parindent}{0.5cm}
\graphicspath{ {C:/Users/manue/Pictures/ImagenesLatex/} }
\begin{document}
\maketitle
\section*{Ejercicio 1}
\noindent Calcula como debes cambiar la escala para trabajar en una hoja
con la unidad en cm.
Para poder cambiar la escala en una hoja tenemos que utiizar el comando \emph{scala},
 con los valores en \textsl{Notación Polaca Inversa (RPN)}. Por ejemplo para cambiar la escala a pulgadas tenemos que
 escribir \emph{72 72 scale} ya que como nos explica el libro de texto de la asignatura donde nos dice que la unidad de medida por
  defecto es el punto Adobe. El punto Adobe
 tiene una medida exacta respecto las pulgadas inglesas, inch (se expresan
 con \emph{"}) que es ésta:
\[1 Adobe point = 1/72 \hspace{0.1cm} inch.\]
\begin{quote}    
 Si tenemos que encontrar la relación que existe entre un centímetro y 
 una pulgada, lo podemos lograr realizando la división del factor de la escala de punto Adobe
  en pulgadas entre el factor de escala de pulgadas a centímetros.
    \[ 72 \div 2,54 = 28,34645669 \]
    Este resultado que por aproximación establecemos en 28,35 es el factor de escala que necesitamos para 
    cambiar la escala de punto Adobe a centímetros.
\end{quote}
¿Que dimensión tiene un folio \emph{A4} en puntos \emph{Adobe}?
\indent Para poder calcular la dimensión de un folio \emph{A4} 
en puntos \emph{Adobe} tenemos que saber que la dimensión de un folio \emph{A4}
 es de 210x297 mm.
 \begin{quote}
    Si queremos saber la dimensión de un folio \emph{A4} 
    en puntos \emph{Adobe} tenemos que realizar la siguiente operación:
    \[ 210 \times 28,35 = 595,5 \]
    \[ 297 \times 28,35 = 841,5 \]
    Donde podemos ver que se realiza la operación de multiplicar 
    las dimensiones de un folio \emph{A4} en milímetros por el factor de escala 
    que hemos determinado en la respuesta anterior.\\
    Por lo tanto la dimensión de un folio \emph{A4} en puntos \emph{Adobe} es de 595,5x841,5.
 \end{quote}

\noindent¿Y un \emph{A3}?
\indent Para poder calcular la dimensión de un folio \emph{A3} en puntos \emph{Adobe} tenemos que saber que la dimensión de un folio \emph{A3}
 es de 297x420 mm.
 \begin{quote}
    Si queremos saber la dimensión de un folio \emph{A3} 
    en puntos \emph{Adobe} tenemos que realizar la misma operación que hemos precisado 
    anteriormente:
    \[ 297 \times 28,35 = 841,5 \]
    \[ 420 \times 28,35 = 1190,5 \]
    Por lo tanto la dimensión de un folio \emph{A3} 
    en puntos \emph{Adobe} es de 841,5x1190,5.
 \end{quote}

 \section*{Ejercicio 7}
 Demuestra que la cantidad $\frac{-C}{\sqrt{A^2+B^2}}$
  es la distancia con signo desde
 la recta $Ax + By = 0$ a la recta $Ax + By + C = 0$.
 Sugerencia: Mira el dibujo y calcula la distancia con signo s de
 la proyección de \emph{u} sobre $[A, B]$
   \begin{figure}[h]
      \centering
      \includegraphics[width=0.5\textwidth]{ejer6}
      \caption{Dibujo del ejercicio 6}
      \label{fig:ejercicio6}
   \end{figure}
\begin{quote}
   Para entender por qué la expresión \(\frac{-C}{\sqrt{A^2 + B^2}}\) 
   representa la distancia con signo de la línea \(Ax + By + C = 0\) 
   de la línea \(Ax + By = 0\), podemos utilizar el concepto de proyecciones 
   y considerar la distancia perpendicular entre las dos líneas.
   
   Primero, observemos que la línea \(Ax + By = 0\)
    es simplemente la misma línea \(Ax + By + 0 = 0\), 
    pero que pasa a través del origen $(0,0)$ y no hay distancia $C$. 
    Entonces, la distancia entre estas dos líneas es básicamente 
    la distancia entre la línea \(Ax + By + C = 0\) y el origen.
   
   Para encontrar esta distancia, podemos proyectar el vector 
   que une un punto \textcolor{red}{$P$} en la línea \(Ax + By + C = 0\)  
   sobre el vector perpendicular de la línea \(Ax + By = 0\) que pasa
   por la línea  \(Ax + By + C = 0 \) que es \([A, B]\), que nos calculará la 
   distancia \textcolor{green}{$s$} entre ambas líneas .
   
   La proyección del vector \([A, B]\) sobre sí mismo es su magnitud:
   
   \[||[A, B]|| = \sqrt{A^2 + B^2}\]
   
   La proyección del vector en el punto sobre la línea \(Ax + By + C = 0\) 
   sobre el vector normal \([A, B]\) es el producto escalar (punto) 
   entre los dos vectores, dividido por la magnitud de \([A, B]\):
   
   \[\frac{-C \bullet [A, B]}{\sqrt{A^2 + B^2}} = \frac{-C}{\sqrt{A^2 + B^2}}\]
   
   Dado que esta es la proyección del vector \textcolor{red}{$\vec{u}$} 
   sobre el vector \([A, B]\), 
   que es la distancia con signo entre la línea \(Ax + By + C = 0\)
    y el origen, se observa que \(\frac{-C}{\sqrt{A^2 + B^2}}\) 
    representa la distancia con signo entre las dos líneas.

   Para poder comprender porqué hablamos de signo en esta distancia entre las
   dos líneas consideradas, hemos de observar que
   la longitud con signo en valor absoluto equivale a la longitud 
   de \emph{\textcolor{green}{s}} en el dibujo
     y es positiva cuando la proyección coincide con
la dirección $[A,B]$, y negativa cuando se proyecta en sentido contrario.
 Queda perfectamente determinada mediante:
 $s = \vec{|[A,B]|} \cos (\alpha )$.

Ahora bien si ponemos $\cos(\alpha)$ en función del vector [A,B] que denominaremos 
para el desarrollo de la explicación como un vector $\vec{v}$ y el vector $\vec{u}$  tenemos:
$$s = \vec{|u|} \cos (\alpha ) = \vec{|u|}\frac{\vec{u}
 \bullet \vec{v}}{\vec{|u|} \vec{|v|}} = \frac{ax + by}{\vec{|v|}}$$

 Y según la explicación de la asignatura ahora 
 podemos poner a \textcolor{green}{$s$} en función de [A,B] como 
\[ s = \frac{{s \mathbf{v}}}{{\|\mathbf{v}\|}} 
= \frac{{x\mathbf{a} + y\mathbf{b}}}{{\|\mathbf{v}\|}} 
= \frac{{x\mathbf{a} + y\mathbf{b}}}{{\|\mathbf{v}\|^2}}[a, b] \]

Para determinar el signo de la distancia hemos de saber si el ángulo ($\theta$) es agudo, 
obtuso o recto. Si el ángulo ($\theta$) es agudo, la proyección es positiva,
 si el ángulo ($\theta$) es obtuso, 
la proyección es negativa y si el ángulo ($\theta$) es recto, la proyección es cero. Para saber 
si el ángulo ($\theta$) es agudo, obtuso o recto, 
hemos visto que se debe utilizar el producto escalar 
de los vectores $\vec{u}$ y $\vec{v}$, que está definido
como el producto de los módulos de ambos vectores por el coseno del 
ángulo menor ($\theta$)
que forman sus direcciones, por lo que siempre tendremos en cuenta a $\pi$ como 
máxima medida del ángulo mediante $cos(\pi - \theta)$.
  
\end{quote} 


   \noindent Por ejemplo en el dibujo que nos consulta el problema, vemos que 
   la proyección del vector $\vec{u}$ $(6,2)$
    sobre el vector [A,B] $(6,8)$ al que denominaremos $\vec{v}$ se puede calcular 
   utilizando la fórmula de la proyección escalar vista anteriormente,
    que es:

   $$\text{proy}_{\mathbf{v}} \mathbf{u} = \frac{\mathbf{u} \bullet \mathbf{v}}{\|\mathbf{v}\|^2} \mathbf{v}$$
   

   Donde:

    --- $\vec{u}$ y $\vec{v}$ son los vectores.

    --- $\vec{u} \bullet \vec{v}$ es el producto escalar de $\vec{u}$ y $\vec{v}$.

    --- $\vec{|v^2|}$ es el cuadrado de la magnitud (o longitud) 
   del vector $\vec{v}$.
   
   Primero, calculamos el producto escalar de $\vec{u}$ y $\vec{v}$:
   
   $$\mathbf{u} \bullet \mathbf{v} = u_1 \bullet v_1 + u_2 \bullet v_2 = 6 \bullet 6 + 2 \bullet 8 = 36 + 16 = 52$$
   
   Luego, calculamos el cuadrado de la magnitud del vector $\vec{v}$
   
   $$\|\mathbf{v}\|^2 = v_1^2 + v_2^2 = 6^2 + 8^2 = 36 + 64 = 100$$
   
   Finalmente, sustituimos estos valores en la fórmula de la proyección para obtener el vector de proyección:
   
   $$\text{proy}_{\mathbf{v}} \mathbf{u} = 
   \frac{52}{100} \mathbf{v} = \left(\frac{52}{100} \cdot 6, \frac{52}{100} \cdot 8\right) = (3.12, 4.16)$$
   
   Por lo tanto, la proyección del vector $\vec{u} = (6,2)$ sobre el
    vector $\vec{v}= (6,8)$ es aproximadamente (3.12 , 4.16).
     \begin{figure*}[h]
      \centering
      \includegraphics[width=0.5\textwidth]{proyeccion2}
      \caption{Proyección del vector}
      \label{fig:ejercicio6_2}
     \end{figure*}
      \begin{quote}
         Que nos da el punto de intersección del segmento \textcolor{red}{$\vec{v}$} con signo \emph{$+$}
         o \emph{$-$} según que el ángulo $\theta$ sea agudo u obtuso, que en este caso 
         al observar el dibujo del problema será agudo (Figura 2) y la distancia será un valor positivo.
      \end{quote}
    



 \section*{Ejercicio 8}
 Dada la línea (recta) $Ax + By + C = 0$ y un punto $P$,
  encuentra en Internet o en un libro de geometria analítica
   la formula para la proyección perpendicular de P en la línea. 
   Pon un ejemplo.

   \noindent La fórmula para encontrar la proyección perpendicular de un punto $ P $ 
   en una línea $ Ax + By + C = 0 $ en geometría analítica es la siguiente:

   

Dado el punto $ P(x_0, y_0) $, la proyección perpendicular $ Q $ de $ P $ en la línea se puede encontrar de la siguiente manera:
\begin{quote}
1. Calculamos la pendiente de la recta perpendicular a la línea dada. 
La pendiente de una recta perpendicular es el negativo del inverso 
de la pendiente de la línea dada. 
Entonces, la pendiente de la recta perpendicular es $ -\frac{B}{A} $.

2. Utilizamos la ecuación punto-pendiente para encontrar la ecuación de la recta
 perpendicular que pasa por el punto $ P(x_0, y_0) $. 
 La ecuación punto-pendiente es $ y - y_0 = m(x - x_0) $, 
 donde $ m $ es la pendiente de la recta. 
 Sustituye $ m = -\frac{B}{A} $ y también $ (x_0, y_0) $ por las coordenadas del punto $ P $.

3. Resuelvemos finalmente el sistema de ecuaciones formado por la ecuación
 de la línea original y la ecuación de la recta perpendicular 
 para encontrar las coordenadas del punto de intersección $ Q $.
  Esto se puede hacer sustituyendo la ecuación de la recta perpendicular
   en la ecuación de la línea original y resolviendo para $ x $ y $ y $.
\end{quote}
Por ejemplo, consideremos la línea $ 2x + 3y - 6 = 0 $ y el punto $ P(4, 2) $.
\begin{quote}
\begin{enumerate}
\item  La pendiente de la recta perpendicular es $-\frac{B}{A} = -\frac{3}{2}$.
\item  La ecuación de la recta perpendicular es $ y - 2 = -\frac{3}{2}(x - 4) $.
\item Sustituyendo la ecuación de la recta perpendicular en la ecuación de la línea original, obtenemos:
\begin{equation} \label{eu_eqn}
2x + 3\left(-\frac{3}{2}(x - 4) + 2\right) - 6 = 0
\end{equation}
\end{enumerate}
\end{quote}
Resolviendo esta ecuación, encontramos las coordenadas del punto de intersección $ Q $, que es la proyección perpendicular de $ P $ en la línea.
\begin{quote}
la ecuación que obtuvimos al sustituir la ecuación de la recta perpendicular
 en la ecuación de la línea original:

\[ 2x + 3\left(-\frac{3}{2}(x - 4) + 2\right) - 6 = 0\;\;\;\;(\ref{eu_eqn}) \] 

\noindent Se resuelve de la siguiente manera:\\
Primero, distribuyamos el término \(3\) dentro del paréntesis:
\begin{align*}
 &2x + 3\left(-\frac{3}{2}x + 6 - 6\right) - 6 = 0\\ 
 &2x + 3\left(-\frac{3}{2}x\right) - 6 = 0\\
 &2x - \frac{9}{2}x - 6 = 0
\end{align*}

Ahora, combinamos términos semejantes:
\begin{align*}
 &\left(2 - \frac{9}{2}\right)x - 6 = 0\\
 &\left(\frac{4}{2} - \frac{9}{2}\right)x - 6 = 0\\
 &\;\;\left(\frac{-5}{2}\right)x - 6 = 0\\
 &\;\;\;\;\;\;\frac{-5x}{2} - 6 = 0
\end{align*}
Para despejar \(x\), sumamos \(6\) a ambos lados de la ecuación:

\[ \frac{-5x}{2} = 6 \]

Para eliminar el denominador \(2\) en \(x\), multiplicamos ambos lados de la ecuación por \(2\):

\[ -5x = 12 \]

Finalmente, para obtener \(x\), dividimos ambos lados de la ecuación por \(-5\):

\[ x = \frac{12}{-5} \]

\[ x = -\frac{12}{5} \]

Ahora que hemos encontrado el valor de \(x\), 
podemos usarlo para encontrar el valor de \(y\) 
sustituyendo en la ecuación de la recta perpendicular.
 Luego, las coordenadas del punto intersección \(Q\)
  serán \((- \frac{12}{5}, \frac{14}{5})\).
\end{quote} 
\section*{Ejercicio 9}
Dadas dos rectas $A1x + B1y + C1 = 0$ y $A2x + B2y + C2 = 0$
encuentra una formula para el punto de interseccion.

\noindent La fórmula para encontrar el punto de intersección 
de dos rectas en el plano:

Dadas las ecuaciones de las rectas:
\begin{quote}
%\begin{enumerate} 
   \newcounter{enumTemp}
   \newenvironment{enumTemp}[1][]{\refstepcounter{enumTemp}\par\medskip
   \noindent\textbf{Ecuacion~\theenumTemp. #1} \rmfamily}{\medskip}
   \begin{enumTemp}
   \(A_1x + B_1y + C_1 = 0\) \label{1}
   \end{enumTemp}
   \begin{enumTemp}
   \(A_2x + B_2y + C_2 = 0\) \label{2}
   \end{enumTemp}
%\end{enumerate}
\end{quote}

El punto de intersección \( (x, y) \) puede ser encontrado 
usando las siguientes fórmulas:
\begin{equation}
x = \frac{{B_1C_2 - B_2C_1}}{{A_1B_2 - A_2B_1}}
\end{equation}
\begin{equation} 
y = \frac{{A_2C_1 - A_1C_2}}{{A_1B_2 - A_2B_1}}
\end{equation}

Estas dos fórmulas son el resultado de resolver el sistema formado
 por las ecuaciones de las rectas. Sustituyendo \( x \) 
 en cualquiera de las ecuaciones de las rectas, obtenemos \( y \), 
 y viceversa. 
 
1. Despejamos \( y \) en términos de \( x \) en cada ecuación 
(\ref{1})\,(\ref{2}):

   Para la primera ecuación:
   \[ B_1y = -A_1x - C_1 \]
   \[ y = \frac{{-A_1x - C_1}}{{B_1}} \]

   Para la segunda ecuación:
   \[ B_2y = -A_2x - C_2 \]
   \[ y = \frac{{-A_2x - C_2}}{{B_2}} \]

2. Igualamos las dos expresiones para \( y \):

   \[ \frac{{-A_1x - C_1}}{{B_1}} = \frac{{-A_2x - C_2}}{{B_2}} \]

3. resolveremos esta ecuación para encontrar \( x \) y 
una vez que tenemos \( x \), podemos encontrar \( y \) sustituyendo \( x \) 
en cualquiera de las ecuaciones de las rectas.

4. Una vez que hemos encontrado \( x \) e \( y \),
 estas coordenadas formarán el punto de intersección de las dos rectas.

Debemos tener cuidado con los casos en los que las rectas
 pueden ser paralelas o coincidentes,
  ya que en esos casos no habrá un punto de intersección o no será único.

\section*{Ejercicio 10}
Dados dos puntos P y Q, encuentra la ecuacion de la recta que
los contiene.

\noindent La fórmula para encontrar la ecuación de la recta 
que contiene dos puntos \( P(x_1, y_1) \) y \( Q(x_2, y_2) \) 
en el plano es utilizando la fórmula de la pendiente-intersección
 la siguiente:

La ecuación de la recta en su forma más común es 
la forma pendiente-intersección:

\[ y = mx + b \]

Donde:
\begin{quote}
   - \( m \) es la pendiente de la recta.

   -   \( b \) es la intersección en \( y \), 
   es decir, el valor de \( y \) cuando \( x = 0 \), 
   también conocido como la ordenada al origen.
\end{quote}

La pendiente \( m \) de la recta se puede calcular utilizando 
la fórmula de la pendiente, que compara el cambio en \( y \) 
con el cambio en \( x \) entre dos puntos en la recta. 
Si tenemos dos puntos \( P(x_1, y_1) \) y \( Q(x_2, y_2) \), 
la fórmula de la pendiente es:

\[ m = \frac{{y_2 - y_1}}{{x_2 - x_1}} \]

Una vez que hemos calculado la pendiente \( m \), 
podemos usar cualquiera de los dos puntos \( P \) o \( Q \)
 para encontrar el valor \( b \) de la intersección en \( y \).
  Esto se puede hacer sustituyendo las coordenadas \( x \) e \( y \) 
  de uno de los puntos en la ecuación de la recta:

\[ y_1 = mx_1 + b \]

Luego, despejamos \( b \) para encontrar su valor.

Una vez que hemos encontrado \( m \) y \( b \), 
podemos escribir la ecuación de la recta que pasa por
 los puntos \( P \) y \( Q \).


\section*{Ejercicio 11}
Una función afín $f(x, y) = Ax + By + C$ vale $-4$ en el punto
$(0, 0)$ y $7$ en el punto $(1, 2)$.

$$\,\,\,\;\,f(0, 0) = -4$$
$$f(1, 2) = 7$$
En que punto del segmento entre los puntos anteriores tenemos
$f(x, y) = 0$?

\noindent Para encontrar el punto en el segmento entre los puntos dados donde la función $f(x, y)$ es igual a $0$, podemos usar la ecuación de la función afín y resolver para $x$ e $y$.

Dado que la función afín es $f(x, y) = Ax + By + C$, podemos utilizar los puntos dados $(0, 0)$ y $(1, 2)$ para formar un sistema de ecuaciones.

Para $(0, 0)$:
\[f(0, 0) = -4 = A(0) + B(0) + C\]
\[C = -4\]

Para $(1, 2)$:
\[f(1, 2) = 7 = A(1) + B(2) - 4\]
\[A + 2B = 11\]

Entonces, tenemos el sistema de ecuaciones:
\[\begin{cases} A + 2B = 11 \\ C = -4 \end{cases}\]

Resolviendo el sistema con unos valores que lo cumplan, obtenemos los valores de $A$, $B$ y $C$:
\[A = 3, \quad B = 4, \quad C = -4\]

La función afín se convierte en:
\[f(x, y) = 3x + 4y - 4\]

Para encontrar el punto en el que $f(x, y) = 0$, sustituimos en la ecuación y resolvemos:
\[3x + 4y - 4 = 0\]
\[3x + 4y = 4\]

Ahora, para encontrar el punto $(x, y)$ en el segmento entre $(0, 0)$ y $(1, 2)$ donde $f(x, y) = 0$, podemos usar la ecuación paramétrica de la línea que pasa por los dos puntos.

La ecuación paramétrica de una línea que pasa por dos puntos $(x_1, y_1)$ y $(x_2, y_2)$ es:
\[x = x_1 + t(x_2 - x_1)\]
\[y = y_1 + t(y_2 - y_1)\]

Donde $t$ es un parámetro que varía de $0$ a $1$.

Sustituyendo los puntos dados:
\[x = 0 + t(1 - 0) = t\]
\[y = 0 + t(2 - 0) = 2t\]

Sustituimos estas expresiones en la ecuación de $f(x, y)$:
\[3(t) + 4(2t) - 4 = 0\]
\[3t + 8t - 4 = 0\]
\[11t - 4 = 0\]
\[11t = 4\]
\[t = \frac{4}{11}\]

Por lo tanto, el punto donde $f(x, y) = 0$ 
en el segmento entre $(0, 0)$ y $(1, 2)$ es $(\frac{4}{11}, \frac{8}{11})\equiv$ (0.36 , 0.72).




\end{document}
